\input{preamble.tex}

\begin{document}
    \Header

    \BeginConspect

    \Section{Математическая логика}{}{Гагин Артур}

    \Subsection{Пропозициональная формула: ПП, логическая связка, скобочная структура}
    
    \begin{Def}[Пропозициональный алфавит]
        Определим пропозициональный алфавит следующим образом.
        $$\mathcal{A} = \{ x, |, \#, \neg, \&, \vee, \supset, \equiv, (, )  \}.$$
    \end{Def}

    \begin{Def}[Пропозициональная буква]
        Пропозициональную букву (ПБ) определим рекурсивно.
        \begin{itemize}
            \item $x$ --- ПБ.
            \item Если $\epsilon$ --- ПБ, то $\epsilon|$ --- ПБ.
            \item Других ПБ нет.
        \end{itemize}
    \end{Def}

    \begin{Def}[Пропозициональная переменная]
        Пропозициональная переменная (ПП) имеет следующий вид: ПБ\#.
    \end{Def}

    \begin{Rem}
        Под следующими равенствами будем понимать синтаксическое равенство.
        \begin{itemize}
            \item $x_0 = x\#$.
            \item $x_1 = x|\#$.
            \item ...
            \item $x_n = x| ...n_{times}... |\#$.
        \end{itemize}
        При этом, если говорим, что $u$ --- ПП, то подразумеваем, что существует такое $i$, что $u = x_i$ синтаксически.
    \end{Rem}

    \begin{Def}[Пропозициональная формула]
        Определим пропозициональную формулу (ПФ) рекурсивно.
        \begin{itemize}
            \item ПП --- ПФ.
            \item Если $A$ и $B$ --- ПФ, то $(A * B)$ --- ПФ, где $*$ --- один из следующих символов: $\&, \vee, \supset, \equiv$.
            \item Если $A$ --- ПФ, то $(\neg A)$ --- ПФ.
            \item Других ПФ нет.
        \end{itemize}
    \end{Def}

    \begin{Def}[Логическая связка]
        Логической связкой будем называть один из следующих символов: $\neg, \&, \vee, \supset, \equiv$.
    \end{Def}

    \Subsection{Конъюнктивная (дизъюнктивная) нормальная форма. Теорема о существовании КНФ и ДНФ для произвольной формулы}

    \begin{Def}[Элементарная дизъюнкция (конъюнкция)]
        Пусть $x_1, ..., x_n$ --- список ПП. Определим элементарную дизъюнкцию (конъюнкцию) рекурсивно.
        \begin{itemize}
            \item ПП из списка $x_1, ..., x_n$.
            \item Отрицание ($\neg$) ПП из $x_1, ..., x_n$.
            \item Если $X, Y$ --- ЭД (ЭК), то $X \vee Y$ --- ЭД ($X \& Y$ --- ЭК).
            \item Других ЭД (ЭК) нет.
         \end{itemize}
    \end{Def}

    \begin{Def}[Полная элементарная дизъюнкция (конъюнкция)]
        ЭД (ЭК) называется полной, если все ее переменные входят ровно по одному разу.
    \end{Def}

    \begin{Example}
        Пусть $x_1, x_2, x_3$ --- список ПП.
        \begin{itemize}
            \item $x_1$ --- ЭД и ЭК.
            \item $x_1 \vee x_3$ --- ЭД.
            \item $x_2 \& x_3$ --- ЭК.
            \item $x_1 \vee x_2 \vee x_3$ --- ПЭД.
            \item $x_1 \& x_2 \& x_3$ --- ПЭК.
        \end{itemize}
    \end{Example}

    \begin{Def}[Конъюнктивная (дизъюнктивная) нормальная форма]
        ПФ находится в КНФ (ДНФ), если она является либо ЭД (ЭК), либо конъюнкцией (дизъюнкцией) ЭД (ЭК).
    \end{Def}

    \begin{Thm}[Теорема о существовании КНФ и ДНФ для произвольной формулы]
        Любая ПФ формула $A$ может быть приведена к КНФ (ДНФ) $A^*$, такой, что $A \sim A^*$.
    \end{Thm}

    \Subsection{Совершенная конъюнктивная (дизъюнктивная) нормальная форма. Теорема о
существовании СКНФ и СДНФ}

    \begin{Def}[Совершенная конъюнктивная (дизъюнктивная) нормальная форма]
        ПФ находится в СКНФ (СДНФ), если она находится в КНФ (ДНФ), а также каждая ее ЭД (ЭК) является полной.
    \end{Def}

    \begin{Thm}[Теорема о существовании СКНФ и СДНФ]
        Пусть $A$ --- не тождественно ложная ПФ. Тогда существует ПФ $A^*$, которая находится в СКНФ (СДНФ), такая, что $A \sim A^*$.
    \end{Thm}

    \Subsection{Существенные и несущественные переменные пропозициональной формулы. Добавление фиктивных переменных}

    \begin{Def}[Существенные и несущественные переменные пропозициональной формулы]
        ПП $\alpha_i$ пропозициональной формулы $\mathcal{A}(\alpha_1, ..., \alpha_n)$ является существенной, если существует набор значений $\beta_1, ..., \beta_{i-1}, \beta_{i+1}, ..., \beta_n$ для переменных $\alpha_1, ..., \alpha_{i-1}, \alpha_{i+1}, ..., \alpha_n$, такой, что 
        $$\mathcal{A}(\beta_1, ..., \beta_{i-1}, true, \beta_{i+1}, ..., \beta_n) \neq \mathcal{A}(\beta_1, ..., \beta_{i-1}, false, \beta_{i+1}, ..., \beta_n).$$

        ПП называется фиктивной (несущественной), если она не является существенной.
    \end{Def}

    \begin{Rem}[Добавление фиктивных переменных]
        Добавление фиктивных переменных не влияет на значение ПФ при любой возможной интерпретации $\sigma$.
    \end{Rem}

    \Subsection{Валентность ПФ, тавтологии. Теорема о тавтологии. Выражение произвольной ПФ через три логические связки. Законы классической логики}

    \Subsection{Логическое следствие}
    
    \begin{Def}[Валентность ПФ]
        Рассмотрим таблицу истинности ПФ $A$, а точнее ее строку с номером $j$. Будем говорить, что валентность ПФ $A$ на наборе $e_{j1}, ..., e_{jn}$ ($n$ --- количество ПП, входящих в ПФ $A$) равна $\delta_j$ (true/false) и писать 
        $$val_{e_{j1}, ..., e_{jn}}A = \delta_j.$$
    \end{Def}

    \begin{Def}[Интерпретация]
        Пусть $A = A(v_1, ..., v_n)$ --- ПФ. Интерпретацией называется правило (функция), сопоставляющее каждой ПП $v_1, ..., v_n$ значение И или Л.
    \end{Def}

    \begin{Rem}
        Обозначение:
        $$\sigma: \{ \text{ПП} \} \rightarrow \{ \text{И, Л} \}.$$
    \end{Rem}

    \begin{Def}[Истинность формулы при интерпретации]
        Пусть $A = A(v_1, ..., v_n)$ --- ПФ, $\sigma$ --- некоторая интерпретация.

        Определим отношение $\sigma \models A$ (формула истинна при интерпретации $\sigma$) рекурсивно.
        \begin{itemize}
            \item $\sigma \models v$ тогда, и только тогда, когда $\sigma(v) =$ И.
            \item $\sigma \models \neg A$ тогда, и только тогда, когда не выполнено $\sigma \models A$.
            \item $\sigma \models A \& B$ тогда, и только тогда, когда $\sigma \models A$ и $\sigma \models B$.
            \item $\sigma \models A \vee B$ тогда, и только тогда, когда $\sigma \models A$ или $\sigma \models B$.
            \item $\sigma \models A \supset B$ тогда, и только тогда, когда $\sigma \models \neg A \vee B$.
            \item $\sigma \models A \equiv B$ тогда, и только тогда, когда $\sigma \models (A \supset B) \& (B \supset A)$.
        \end{itemize}
    \end{Def}

    \begin{Def}[Выполнимая ПФ]
        ПФ $A$ называется выполнимой, если существует интерпретация $\sigma$, такая, что $\sigma \models A$.
    \end{Def}

    \begin{Def}[Тавтология]
        ПФ $A$ называется тавтологией, если для интерпретации $\sigma$ верно, что $\sigma \models A$.
    \end{Def}

    \begin{Rem}
        Обозначение: $\models A$.
    \end{Rem}

    \begin{Def}[Противоречие]
        ПФ $B$ называется противоречием, если $\models \neg B$.
    \end{Def}

    \begin{Def}[Логическое следствие]
        ПФ $B$ является логическим следствим ПФ $A$, если $\models (A \supset B)$.
    \end{Def}

    \begin{Rem}
        Обозначение: $A \Rightarrow B$.
    \end{Rem}

    \begin{Thm}[О тавтологии]
        Если $\models A, A \Rightarrow B$, то $\models B$.
    \end{Thm}

    \begin{Rem}
        Произвольная ПФ может быть выражена через логические связки $\neg$ и $\vee$. Для этого используются законы классической логики, некоторые из которых рассмотрим далее.
        \begin{itemize}
            \item $\neg (A \vee B) = \neg A \& \neg B$ и $\neg (A \& B) = \neg A \vee \neg B$ --- законы де Моргана.
            \item $A \equiv B = (A \supset B) \& (B \supset A)$.
            \item $A \supset B = \neg A \vee B$.
            \item $\neg \neg A = A$.
            \item $A \supset B = \neg B \supset \neg A$ --- закон контрпозиции.
            \item $A \& (B \vee C) = (A \& B) \vee (A \& C)$ и $A \vee (B \& C) = (A \vee B) \& (A \vee C)$ --- закон дистрибутивности.
        \end{itemize}
    \end{Rem}

    \Subsection{Формула исчисления предикатов: ПП, ПК, ФС, ПС, кванторы, терм, атомарная формула, скобочная структура}

    \begin{Def}[Алфавит языка исчисления предикатов]
        Алфавит языка $\mathcal{L}$ исчисления предикатов содержит следующие символы.
        $$\mathcal{L} = \{ x,c, P, F, |, \#, \forall, \exists, \neg, \vee, \&, \supset, \equiv, (, ) \}.$$
    \end{Def}

    \begin{Def}[Номер]
        Номер --- это слово в алфавите $\{ | \}$.
    \end{Def}

    \begin{Rem}
        Введем следующие обозначения.
        \begin{itemize}
            \item $0 = \Lambda$.
            \item $1 = 0|$.
            \item ...
            \item $n = \widetilde{n + 1} = n|$.
        \end{itemize}
        Волна над числом говорит нам о том, что это единый символ.
    \end{Rem}

    \begin{Def}[Сумма]
        Конкатенацию номеров назовем их суммой и будем обознать эту операцию знаком +.
    \end{Def}

    \begin{Thm}
        Верны следующие утверждения.
        \begin{itemize}
            \item $k + l = l + k$.
            \item $k + (l + m) = (k + l) + m$.
            \item Если $k = l$, то $k + m = l + m$.
        \end{itemize}
    \end{Thm}

    \begin{Def}[<, $\leq$]
        Будем писать $k \leq l$ тогда и только тогда, когда $k$ --- начало слова $l$, то есть $k + m = l$ для некоторого $m$. В этом случае говорят, что $k$ меньше или равно $l$ ($l$ больше или равно $k$). Если при этом $k \neq l$, то пишут $k < l$ и говорят, что $k$ строго меньше $l$ ($l$ строго больше $k$).
    \end{Def}

    \begin{Thm}
        Верны следующие утверждения.
        \begin{itemize}
            \item $k \leq k$.
            \item Если $k \leq l$ и $l \leq k$, то $k = l$.
            \item Если $k \leq l$ и $l \leq m$, то $k \leq m$. 
            \item Если $k < l$, то $k + m < l + m$.
        \end{itemize}
    \end{Thm}

    \begin{Rem}
        Теоремы выше показывают, что класс всех номеров устроен так, как требуется от натуральных чисел в плане счёта, нумерации и порядка.

        Таким образом, $0 < 1 < 2 < 3 < 4 < 5 < ...$.
    \end{Rem}

    \begin{Def}[Предметная переменная]
        Предметной переменной (ПП) называется слово вида $x\text{НОМЕР}\#$.
    \end{Def}

    \begin{Rem}
        Отметим, что ПП в написании совпадает с пропозициональной переменной, однако, содержательная её часть иная.
    \end{Rem}

    \begin{Def}[Предметная константа]
        Предметной константой (ПК) называется слово вида $c\text{НОМЕР}\#$.
    \end{Def}

    \begin{Rem}
        Если $n$ --- номер, то запись $xn\#$ будет сокращена до записи $x_n$. Тоже самое касается и записи $cn\#$, которая заменяется на $c_n$. Более того, для обозначения переменных (констант) мы будем часто использовать буквы, присущие для этого по контексту ($y, z, u, ...$ для переменных, $100, 5, 4, ...$ для констант). В этом случае если мы говорим, что $v$ является переменной, то это означает, что для некоторого номера $n$ имеет место $v = x_n$. Аналогично объясняется и смысл фразы ``$m$ есть константа''.
    \end{Rem}

    \begin{Def}[Предикатный символ]
        Слово вида $P_k^n = \#nPk\#$, где $n$ и $k$ являются номерами, называется предикатным символом (ПС или просто предикат). 
        
        Номер $n$ называется местностью предиката, а $k$ --- это его порядковый номер в классе всех $n$-местных предикатов.
    \end{Def}

    \begin{Rem}
        Мы говорим, что $Q$ --- $n$-местный предикат, если для некоторых $n$, $k$ верно $Q = P^n_k$.
    \end{Rem}

    \begin{Example}
        На данном этапе подобные примеры являются бессмысленными, однако помогают понять, для чего предикатные символы понадобятся далее.
        
        Предикат $EQL(x,y)$ --- двуместный предикат, который равен И, если $x = y$, а в ином случае Л.
    \end{Example}

    \begin{Def}[Функциональный символ]
        Слово вида $F_k^n = \#nFk\#$, где $n$ и $k$ являются номерами, называется функциональным символом (ФС).

         Номер $n$ называется арностью ФС, а $k$ --- это его порядковый номер в классе всех $n$-арных ФС.
    \end{Def}

    \begin{Rem}
        Мы говорим, что $G$ --- $n$-арный ФС, если для некоторых $n$, $k$ верно $G = F^n_k$.
    \end{Rem}

    \begin{Def}[Терм]
        Опредим слово, называемое термом, рекурсивно.
        \begin{itemize}
            \item ПП является термом.
            \item ПК является термом.
            \item если $t_1, ..., t_n$ --- термы, F --- $n$-арный ФС, то $F(t_1, ..., t_n)$ --- терм. 
            \item Других термов нет.
        \end{itemize}
    \end{Def}

    \begin{Rem}
        Пусть $t$ --- терм, а $v_1, ..., v_l$ --- список всех (без повторений) ПП, входящих в терм $t$. В этом случае будем писать $t = t(v_1, v_2, ..., v_l)$.
    \end{Rem}

    \begin{Def}[Атомарная формула]
        Атомарной формулой (АФ) называется слово вида $P (t_1, ..., t_n)$, где $P$ --- $n$-местный ПС, $t_1, ..., t_n$ --- термы.
    \end{Def}

    \begin{Def}[Формула исчисления предикатов]
        Определим слово, называемое формулой исчисления предикатов (ФИП или просто формула), рекурсивно.
        \begin{itemize}
            \item АФ --- ФИП.
            \item Если $A$ --- ФИП, то $\neg A$ --- ФИП.
            \item Если $A$ и $B$ --- ФИП, то слова $(A \vee B)$, $(A \& B)$, $(A \supset B)$, $(A \equiv B)$ --- ФИП.
            \item Если $A$ --- ФИП, а $x$ --- ПП, то слова $(\forall x A)$ и $(\exists x A)$ --- ФИП.
            \item Других ФИП нет.
        \end{itemize}
    \end{Def}

    \Subsection{Свободные и связанные переменные в ФИП, замкнутый терм, замкнутая формула, свобода терма для подстановки}

    \begin{Def}[Зона действия квантора]
        Формула $A$ в ФИП $(\forall x A)$ и $(\exists x A)$ называется зоной действия (вхождения) квантора по переменной $x$.
    \end{Def}

    \begin{Def}[Свободные и связанные переменные ФИП] 
        Вхождение ПП называется связанным, если оно имеет место непосредственно в зоне действия квантора по этой переменной. Вхождение переменной, не являющееся связанным, называется свободным. 
        
        Переменная $x$ называется связанной (свободной) переменной формулы $A$, если она имеет хотя бы одно связанное (свободное) вхождение в эту формулу.
    \end{Def}

    \begin{Def}[Замкнутый терм]
        Терм, не содержащий ни одного вхождения ПП, то есть терм без переменных, называют постоянным или замкнутым.
    \end{Def}
    
    \begin{Def}[Замкнутая формула]
        Формула, не содержащая свободных переменных называется замкнутой формулой или утверждением (высказыванием).
    \end{Def}

    \begin{Def}[Свобода терма для подстановки]
        Пусть $t$ --- терм, $A$ --- формула языка $\mathcal{L}$, $x$ --- ПП. Говорят, что $t$ свободен для подстановки в $A$ вместо $x$, если ни одно свободное вхождение $x$ в $A$ не имеет место в зоне действия квантора по переменной, имеющей вхождение в $t$.

        Более простыми словами, терм свободен для подстановки, если в формулу $[A]^x_t$ все переменные терма $t$ входят свободно.
    \end{Def}

    \begin{Example}
        Пусть $\forall x \varphi(x,y)$ --- некоторая формула. Терм $x$ не является свободным для подстановки вместо $y$. Терм $S(x, y)$ также не является свободным для подстановки вместо $y$. Терм $z$ является свободным для подстановки вместо $y$.
    \end{Example}
    
    \Subsection{Секвенция: список формул, антецедент, сукцедент, аксиомы, правила вывода, вывод в исчислении секвенций, выводимая секвенция, допустимые правила вывода, формульная интерпретация секвенции}

    \begin{Def}[Список формул]
        Под списком формул мы понимаем конечный или пустой набор формул.
    \end{Def}

    \begin{Def}
        Пусть $\Gamma$ --- список формул. 
        \begin{itemize}
            \item $\vee \Gamma$ --- дизъюнкция всех формул из $\Gamma$.
            \item $\neg \Gamma$ --- список формул, который можно получить приписыванием к каждой формуле из $\Gamma$ отрицания.
            \item $\vee$ и $\neg$ пустого списка формул есть Л (ложь).
        \end{itemize}
    \end{Def}

    \begin{Def}[Секвенция, антецедент, сукцедент]
        Секвенция --- выражение вида $\Gamma \rightarrow \Delta$, где $\Gamma$ и $\Delta$ --- cписки формул.

        Запись $\Gamma \rightarrow \Delta$ понимается как сокращение $\Gamma \rightarrow \Delta := \vee(\neg(\Gamma), \Delta)$. Список формул слева от стрелки называется антецедентом, а справа --- сукцедентом.
    \end{Def}

    \begin{Rem}
        Смысл: при допущении всег формул списка $\Gamma$ мы получаем хотя бы одну из формул списка $\Delta$.
    \end{Rem}

    \begin{Def}[Аксиома]
        Секвенция вида $\Gamma_1, A, \Gamma_2 \rightarrow \Delta_1, A, \Delta_2$ называется аксиомой.
    \end{Def}

    \begin{Def}[Правило вывода]
        Правилом вывода называется запись вида $$\frac{S_1, ..., S_n}{T_1, ..., T_k},$$
        где $S_i, T_j (1 \leq i \leq n, 1 \leq j \leq k)$ --- секвенции (первые называются посылками, а последние --- заключениями).
    \end{Def}

    \begin{Rem}
        Смысл: из справедливости посылок следует справедливость заключений.
    \end{Rem}

    \begin{Def}[Вывод]
        Пусть $G = \{ S_1, ..., S_n\}$ --- набор секвенций. Выводом из списка $G$ называется набор секвенций $T_1, ..., T_k$, при этом каждая из секвенций $T_j$
        \begin{itemize}
            \item либо является аксиомой,
            \item либо является одной из секвенций $S_i$,
            \item либо получается из предыдущих при помощи правил вывода.
        \end{itemize}
    \end{Def}

    \begin{Def}[Выводимая секвенция]
        Секвенция $T$ называется выводимой из списка $G$, если существует вывод, такой, что $T$ --- это последняя секвенция этого вывода. 

        Обозначение: $G \vdash T$.

        Секвенция $T$ называется выводимой, если она выводима из пустого списка.

        Обозначение: $\vdash T$.
    \end{Def}

    \begin{Def}[Выводимая формула]
        Формула $A$ выводима из списка формул $A_1, ..., A_n$, если выводима секвенция $A_1, ..., A_n \rightarrow A$.
        
        Формула называется выводимой, если она выводима из пустого списка.
        
        Соответствующие факты будем записывать $A_1, ..., A_n \vdash A$ и $\vdash A$.
    \end{Def}
    
    \begin{Def}[Допустимое правило вывода]
        Правило вывода называется допустимым, если при его добавлении к существующим правилам вывода, класс выводимых секвенций не изменяется.
    \end{Def}

    \begin{Def}
        Формульным образом (интерпретацией) секвенции $\Gamma \rightarrow \Delta$ называется формула $\Phi = (\& \Gamma) \supset (\vee \Delta)$.
    \end{Def}
    
    \Subsection{Теории в языке исчисления предикатов, выводимая формула, присоединение следствий, противоречивая и непротиворечивая теория}

    \begin{Def}[Теория]
        Теорией $T$ называется класс формул.
    \end{Def}

    \begin{Def}[Выводимая формула]
        Формула $A$ называется выводимой из теории $T$, если существует конечная часть $T'$ в $T$, такая, что выводима секвенция $T' \rightarrow A$.
    \end{Def}

    \begin{Rem}
        Обозначение: $T \vdash A$.

        Если $T$ --- пустая теория, то $\vdash A$.
    \end{Rem}

    \begin{Def}[Присоединение следствий]
        Операция присоединения следствий --- это теория $\mathfrak{G}(T)$, получающаяся доавлением к теории $T$ всех ее теорем.

        Иными словами, $A \in \mathfrak{G}(T)$ тогда, и только тогда, когда $T \vdash A$. 
    \end{Def}

    \begin{Def}[Противоречивая и непротиворечивая теория]
        Теория называется противоречивой, если существует формула $A$, такая, что $T \vdash (A \& \neg A)$.

        Теория, не являющаяся противоречивой, называется непротиворечивой.
    \end{Def}

    \begin{Rem}[Непротиворечивая теория]
        Для всякой непротиворечивой теории $T$ существует максимально непротиворечивая теория $T'$, являющаяся расширением $T$, то есть $T \subset T'$, при этом $T'$ непротиворечива, а также для любой непротиворечивой теории $T_1$, которая также является расширением $T$, верно, что $T_1 \subseteq T'$.
    \end{Rem}

    \Subsection{Алгебраическая структура для языка исчисления предикатов: имя предиката в алгебраической структуре, имя константы в алгебраической структуре, имя функтора в алгебраической структуре}

    \begin{Def}[Алгебраическая структура]
        Пусть $M$ --- некоторое множество, $\mathfrak{F}$ --- множество алгебраических операций на $M$ (то есть $* \in \mathfrak{F}$ тогда и только тогда, когда $*$ --- есть функция с областью определения $M^n = M \times M \times ... \times M$ (для некоторого $n$), принимающая значения в $M$), а $\mathfrak{B}$ --- множество отношений на $M$ (отношением, точнее $n$-местным отношением, называется произвольное подмножество $R \subset M^n$). 

        Тогда тройка $\mathfrak{M} = <M, \mathfrak{F}, \mathfrak{B}>$ называется алгебраической структурой.
    \end{Def}

    \begin{Def}
        Пусть $L_1$ --- формализованный язык исчисления предикатов с равенством, $\mathfrak{M}$ --- некоторая алгебраическая структура. Определим функцию $\Pi$ следующим образом.
        \begin{itemize}
            \item Для каждой константы $c$ языка $L_1$ положим $\Pi(c)$ --- некоторый элемент $M$.
            \item Для каждого функционального символа $F$ положим $\Pi(F)$ --- алгебраическая операция на $M$ соответствующей арности.
            \item Для каждого предикатного символа $P$ положим $\Pi(P)$ --- отношение на $M$ соответствующей арности.
        \end{itemize}
        Таким образом, для каждого специального символа $a$ языка $L_1$ определено его имя $\Pi(a)$ в алгебраической структуре. В дальнейшем будем писать $a^{\mathfrak{M}}$.
    \end{Def}

    \Subsection{Интерпретация ЯИП в алгебраической структуре: значение терма в алгебраической структуре, истинность ФИП в алгебраической структуре, выполнимость ФИП, понятие модели}

    \begin{Def}[Интерпретация ЯИП в алгебраической структуре]
        Функция $s$, заданная на множестве всех (предметных) переменных языка $L_1$ и принимающая значения в $M$ называется интерпретацией $L_1$ в $M$.
    \end{Def}

    \begin{Def}[Значение терма в алгебраической структуре]
        Для каждого терма $t$ в $L_1$ и интерпретации $s$  определим значение терма $t$ в структуре $\mathfrak{M}$ при интерпретации $s$, которое будем обозначать $t^{\mathfrak{M}}[s]$.
        \begin{itemize}
            \item Если $t = x$ --- ПП, то $t^{\mathfrak{M}}[s] = s(x)$.
            \item Если $t = c$ --- ПК, то $t^{\mathfrak{M}}[s] = c^{\mathfrak{M}}$. 
            \item Если $t_1, ..., t_2$ --- термы, а $t = F(t_1, ..., t_n)$ --- ФС, то $t^{\mathfrak{M}}[s] = F^{\mathfrak{M}}(t_1^{\mathfrak{M}}[s], ..., t_n^{\mathfrak{M}}[s])$.
        \end{itemize}
    \end{Def}

    \begin{Example}
        Пусть $L(\theta, e, \epsilon, P, S)$ --- язык алгебраических выражений с соответствующей структурой, где $\theta, e$ --- ПК, $\epsilon$ --- ПС, $P, S$ --- ФС. 

        Пусть $\mathfrak{N} = <\N, +, *, =>$ --- алгебраическая структура. 

        Свяжем ее с языком, выдав ПК, ПС и ФС имена: 

        $$\theta^{\mathfrak{N}} = 0, e^{\mathfrak{N}} = 1, \epsilon^{\mathfrak{N}} = \ =, P^{\mathfrak{N}} = +, S^{\mathfrak{N}} = *.$$

        Рассмотрим следующие интерпретации.
        $$s_1(x) = s_1(y) = 1; s_2(x) = 2, s_2(y) = 0.$$

        Также рассмотрим терм $$t_1(x,y) := S(P(x,y), e).$$

        Распишем этот терм в алгебраической структуре $\mathfrak{N}$ при интерпретации $s_1$.

        $$t_1^{\mathfrak{N}}[s_1] := S^{\mathfrak{N}}(P^{\mathfrak{N}}(s_1(x), s_1(y)), e^{\mathfrak{N}}).$$
        
        $$t_1^{\mathfrak{N}}[s_1] := +(*(1,1), 1) = (1 * 1) + 1 = 2.$$

        Далее при интерпретации $s_2$.
        $$t_1^{\mathfrak{N}}[s_1] := +(*(s_2(x),s_2(y)), 1) = (2 * 0) + 1 = 1.$$
    \end{Example}

    \begin{Def}[Истинность ФИП в алгебраической структуре]
        Пусть $L_1$ --- ЯИП, $\mathfrak{M}$ --- алгебраическая структура, $\varphi(x_1, ..., x_n)$ --- ФИП, $s$ --- интерпретация $L_1$ в $\mathfrak{M}$.

        Определим соответствие $\mathfrak{M} \models \varphi[s]$ рекурсивно по определению ФИП.
        \begin{itemize}
            \item Если $\varphi = P(t_1, ..., t_m)$ --- атомарная формула (АФ), то $\mathfrak{M} \models \varphi(s)$ тогда и только тогда, когда $<t_1^{\mathfrak{M}}[s], ..., t_m^{\mathfrak{M}}[s]> \in P^{\mathfrak{M}}$ ($P^{\mathfrak{M}}$, в свою очередь, принадлежит множеству отношений, которое мы обозначили $\mathfrak{B}$) [предположительно, можно выразиться также следующим образом: $P^{\mathfrak{M}}(t_1^{\mathfrak{M}}[s], ..., t_m^{\mathfrak{M}}[s])$ истинна]. 
            \item Если $\varphi = \neg \varphi_1$, то $\mathfrak{M} \models \varphi[s]$ тогда и только тогда, когда не $\mathfrak{M} \models \varphi_1[s]$.
            \item Если $\varphi = \varphi_1 \vee \varphi_2$, то $\mathfrak{M} \models \varphi[s]$ тогда и только тогда, когда $\mathfrak{M} \models \varphi_1[s]$ или $\mathfrak{M} \models \varphi_2[s]$.
            \item ...
            \item Если $\varphi = \exists x \varphi_1(x, x_1, ..., x_n)$, то $\mathfrak{M} \models \varphi[s]$ тогда и только тогда, когда найдется элемент $m \in M$, такой, что $$\mathfrak{M} \models \varphi_1(x_1, ..., x_n)[s \binom{x}{m}],$$ где $s \binom{x}{m}$ --- это такая интерпретация $s_1$, что $s_1(x) = m$ и $s_1(y) = y$ для любого $y \neq x$.
            \item Если $\varphi = \forall x \varphi_1(x, x_1, ..., x_n)$, то $\mathfrak{M} \models \varphi[s]$ тогда и только тогда, когда $\mathfrak{M} \models \neg \exists x \neg \varphi_1[s]$.
        \end{itemize}
    \end{Def}

    \begin{Def}[Выполнимость ФИП и понятие модели]
        Теория $T$ называется выполнимой, если существует алгебраическая структура $\mathfrak{M}$ такая, что для каждого утверждения $\alpha \in T \ \mathfrak{M} \models \alpha$. В этом случае говорят, что $\mathfrak{M}$ является моделью теории $T$.
    \end{Def}
    
    \Subsection{Теорема о значении терма}

    \begin{Thm}[О значении терма]
        Пусть $L_1$ --- ЯИП, $\mathfrak{M}$ --- алгебраическая структура, $s_1, s_2$ --- интерпретации $L_1$ в $\mathfrak{M}$, $t = t(x_1, ..., x_n)$ --- терм в $L_1$.

        Если $s_1(x_i) = s_2(x_i) (1 \leq i \leq n)$, то $t^{\mathfrak{M}}[s_1] = t^{\mathfrak{M}}[s_2]$.
    \end{Thm}

    \Subsection{Теорема об истинности}

    \begin{Thm}[Об истинности]
        Пусть $L_1$ --- ЯИП, $\mathfrak{M}$ --- алгебраическая структура, $\varphi(x_1, ..., x_n)$ --- ФИП, $s_1, s_2$ --- интерпретации $L_1$ в $\mathfrak{M}$, которые имеют одинаковые значения на всех свободных переменных формулы $\varphi$. Тогда $\mathfrak{M} \models \varphi[s_1]$ тогда и только тогда, когда $\mathfrak{M} \models \varphi[s_2]$.
    \end{Thm}

    \Subsection{Теорема о семантическом обосновании исчисления секвенций}

    с.106 или 54?

    \Subsection{Непротиворечивость теории, имеющей модель}

    \begin{Thm}
        Если теория выполнима (имеет модель), то она непротиворечива. Верно и обратное, всякая непротиворечивая теория имеет модель (теорема Геделя о полноте). 
    \end{Thm}

    \Subsection{Теория равенства}

    \begin{Def}[Теория равенства]
        Теорией равенства называется теория, которая имеет язык $\mathfrak{L}_=$ (его сигнатура содержит единственный двухместный предикат =).
        
        Этот предикат называется предикатом равенства. Для него выполнены следующие аксиомы (равенства).
        \begin{itemize}
            \item $\forall x (x = x)$.
            \item $\forall x \forall y ((x = y) \supset (y = x))$.
            \item $\forall x \forall y \forall z ((x = y) \& (y = z) \supset (x = z))$.
        \end{itemize}
    \end{Def}

    \Subsection{Аксиомы согласования с равенством}

    \begin{Def}[Теория с равенством, аксиомы согласования с равенством]
        Пусть $T$ --- некоторая теория в языке $\mathfrak{L}$. Пусть в сигнатуре данной теории выделен двухместный предикатный символ =, для которого в $T$ входят аксиомы равенства. 
        
        Предположим, что для любого ПС $P$ и для любого ФС $F$ в $\mathfrak{L}_T$ справедливы следующие аксиомы, которые называются аксиомами согласования с равенством.
        \begin{itemize}
            \item $\forall x_1 ... \forall x_l \forall y_1 ... \forall y_l (x_1 = y_1 \& ... \& x_l = y_l \supset (P(x_1, ..., x_l) \equiv P(y_1, ..., y_l)))$.
            \item $\forall x_1 ... \forall x_l \forall y_1 ... \forall y_l (x_1 = y_1 \& ... \& x_l = y_l \supset (F(x_1, ..., x_l) = F(y_1, ..., y_l)))$.
        \end{itemize}
        Тогда теория называется теорией с равенством.
    \end{Def}

    \Subsection{Теория групп}

    \begin{Def}[Теория групп]
        Теория групп $T_G$ в сигнатуре языка помимо ПС = имеет ещё ПК 1 и ФС $\times$. Кроме аксиом равенства и согласования с равенством собственными аксиомами теории групп являются следующие.
        \begin{itemize}
            \item $\forall x \forall y \forall z (x \times (y \times z) = (x \times y) \times z)$.
            \item $\forall x (x \times 1 = x)$.
            \item $\forall x \exists y (x \times y = 1)$.
        \end{itemize}
    \end{Def}

    \Subsection{Теория порядка}

    \begin{Def}[Теория порядка]
        Теория порядка --- теория с равенством, язык которой содержит двухместный предикатный символ $\leq$, а также выполнены следующие аксиомы (порядка).
        \begin{itemize}
            \item $\forall x (x \leq x)$.
            \item $\forall x \forall y ((x \leq y) \& (y \leq x) \supset (x = y))$.
            \item $\forall x \forall y \forall z ((x \leq y) \& (y \leq z) \supset (x \leq z))$.
        \end{itemize}
    \end{Def}

    \Subsection{Парадокс Рассела}

    \begin{Rem}[Парадокс Рассела]
        Рассмотрим множество $R := \{x : x \notin x \}$. 

        Верно ли, что $R \in R$? 
        
        Если $R \in R$, то так как $R$ обладает свойством, определяющим $R$, то  $R \notin R$. 
        
        Но если $R \notin R$, то $R$ обладает свойством, определяющим $R$ и, следовательно, $R \in R$. 
        
        Таким образом, $R \in R$ тогда и только тогда, когда $R \notin R$, что и является противоречием, называемым парадоксом Рассела.
    \end{Rem}

    \Subsection{Алфавит. Слово в алфавите. Язык над алфавитом. Равенство слов. Конкатенация слов. Подслово. Интервал вхождения. Подстановка}

    \begin{Def}[Алфавит]
        Алфавит --- конечное множество символов.
    \end{Def}
    
    \begin{Def}[Язык]
        Язык --- множество слов в алфавите $\mathcal{A}$.
    \end{Def}

    \begin{Def}[Слово]
        Слово в алфавите $\mathcal{A}$ --- последовательность символов из $\mathcal{A}$.

        \begin{itemize}
            \item $\Lambda$ --- слово в языке $L$ (в алфавите $\mathcal{A}$), где $\Lambda$ --- пустое слово (слово, которое не содержит ни одного символа).
            \item Если $X$ --- слово, $\alpha$ --- буква, то $X\alpha$ --- слово.
            \item Других слов нет.
        \end{itemize}
    \end{Def}

    \begin{Def}[Равенство слов]
        Слова называются равными, если они равны графически, то есть состоят из одинаковых символов, которые идут в одинаковом порядке.
    \end{Def}

    \begin{Def}[Длина слова]
        Длина слова --- количество символов (букв) в нем.
        \begin{itemize}
            \item $|\Lambda| = 0$.
            \item $|X\alpha| = |X| + 1$.
        \end{itemize}
    \end{Def}

    \begin{Def}[Конкатенация слов]
        Конкатенация слов $A$ и $B$ (приписывание одного слова к другому) --- операция, которую обозначим $AB$ и определим следующим образом.
        \begin{itemize}
            \item Если $B = \Lambda$, то $A B = A$. 
            \item Если $B = \beta B'$, то $A B = A \beta B'$ --- конкатенация слов $A \beta$ и $B'$.
        \end{itemize}
    \end{Def}
    
    \begin{Def}[Вхождение в слово (подслово)]
        Слово $A$ входит в слово $B$ (является его подсловом), если существуют слова $\Delta_1$ и $\Delta_2$ (возможно и пустые), что $$B = \Delta_1 A \Delta_2.$$
    \end{Def}

    \begin{Rem}
        Вхождение слова $X$ в слово $\theta$ будем обозначать $\Sigma = \theta_1 \downarrow X \downarrow \theta_2$ ($\downarrow$ --- не является символом алфавита).
    \end{Rem}

    \begin{Rem}
        Очевидно, что вхождения слова $A$ в слово $B$ могут быть упорядочены отношением ``быть левее''.
    \end{Rem}

    \begin{Example}
        Пусть $B = abcabcabcabca, A = cabc$.
        \begin{itemize}
            \item $\Sigma_1 = ab \downarrow cabc \downarrow abcabca$.
            \item $\Sigma_2 = abcab \downarrow cabc \downarrow abca$.
            \item $\Sigma_3 = abcabcab \downarrow cabc \downarrow a$.
        \end{itemize}
    \end{Example}
    
    \begin{Def}[Интервал вхождения]
        Пусть $\Sigma = \theta_1 \downarrow A \downarrow \theta_2$ --- вхождение слова $A$ в слово $B, k = |\theta_1|, l = |\theta_1 A|$.

        Тогда $(k, l)$ --- интервал вхождения (в $\N$).
    \end{Def}

    \begin{Def}[Подстановка]
        Пусть слово $A$ входит в слово $B$, $C$ --- любое слово в языке $L$.

        Результатом подстановки слова $C$ вместо $k$-ого вхождения слова $A$ в слово $B$ называется слово 
        $$[B_k]^A_C,$$ которое получается заменой $k$-ого вхождения слова $A$ на слово $C$. 
    \end{Def}

    \begin{Rem}
        $[B_{kl}]^{A_1 A_2}_{C_1 C_2} = [[B_k]^{A_1}_{C_1 \ l}]^{A_2}_{C_2}$.
    \end{Rem}
\end{document}